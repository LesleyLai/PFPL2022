\documentclass[12pt]{article}
\setlength{\oddsidemargin}{0in}
\setlength{\evensidemargin}{0in}
\setlength{\textwidth}{6.5in}
\setlength{\parindent}{0in}
\setlength{\parskip}{\baselineskip}

\usepackage{amsmath,amsthm,amsfonts,amssymb,fancyvrb}
\usepackage{mathpartir}

% Helpful macros
\newcommand{\mt}[1]{\ensuremath{\text{#1}}}
\newcommand{\isA}[2]{\ensuremath{#1 \; #2}}
\newcommand{\isANat}[1]{\isA{#1}{\mt{nat}}}
\newcommand{\isABinary}[1]{\isA{#1}{\mt{binary}}}
\newcommand{\zero}{\mt{zero}}
\newcommand{\mySucc}[1]{\mt{succ}(#1)}

\newcommand{\twice}{\mt{twice}}
\newcommand{\twiceOne}{\mt{twice+1}}


%% Judgments


\begin{document}

PFPL \hfill Chapter 2 Exercise\\
Lesley Lai

\hrulefill

\subsection*{Questions}

\begin{enumerate}
\item[2.1] Give an inductive definition of the judgment $\text{max}(m;n;p)$ where $\isANat{m}, \isANat{n}, \isANat{p}$, with the meaning that $p$ is the larger of $m$ and $n$. Prove that every $m$ and $n$ are related to a unique $p$ by this judgment.\\

\begin{mathpar}
\inferrule{
}{
    \text{max}(m;\zero;m)
}

\inferrule{
}{
    \text{max}(\zero;\mySucc{n};\mySucc{n})
}

\inferrule{
    \text{max}(m;n;p)
}{
    \text{max}(\mySucc{m};\mySucc{n};\mySucc{p})
}
\end{mathpar}

\begin{proof}
\textit{If $n = 0$, then we have a unique $p = m$. Similarly, if $m = 0$, then $p = n$. Otherwise we need to prove that if every $m$ and $n$ are related to a unique $p$, then every every \mySucc{m} and \mySucc{n} are related to a unique $q$, and by definition we know that $q = \mySucc{p}$. Thus, by rule induction, every $m$ and $n$ are related to a unique $p$.}
\end{proof}

\item[2.5] Give an inductive definition of the \textit{binary natural numbers}, which are either zero, twice a binary number, or one more than twice a binary number. The size of such a representation is logarithmic, rather than linear, in the natural number it represents.

\begin{mathpar}
\inferrule{
}{
    \isABinary{\zero}
}

\inferrule{
    \isABinary{x}
}{
    \isABinary{\twice(x)}
}

\inferrule{
    \isABinary{x}
}{
    \isABinary{\twiceOne(x)}
}
\end{mathpar}


\item[2.6] Give an inductive definition of addition of binary natural numbers as defined in Exercise \textbf{2.5}.\\ \textit{Hint:} Proceed by analyzing both arguments to the addition, and make use of an auxiliary function to compute the successor of a binary number. Hint: Alternatively, define both the sum and the sum-plus-one of two binary numbers mutually recursively.

\begin{mathpar}
\inferrule{
}{
    \text{succ}(\zero;\twiceOne(\zero))
}

\inferrule{   
}{
    \text{succ}(\twice(n);\twiceOne(n))
}

\inferrule{
    \text{succ}(n;m)
}{
    \text{succ}(\twiceOne(n);\twice(m))
}
\end{mathpar}

\begin{mathpar}
\inferrule{
}{
    \text{add}(n;\zero;n)
}

\inferrule{
}{
    \text{add}(\zero;n;n)
}\\

\inferrule{
    \text{add}(m;n;p)
}{
    \text{add}(\twice(m);\twice(n);\twice(p))
}\\

\inferrule{
    \text{add}(m;n;p)
}{
    \text{add}(\twice(m);\twiceOne(n);\twiceOne(p))
}\\

\inferrule{
    \text{add}(m;n;p)
}{
    \text{add}(\twiceOne(m);\twice(n);\twiceOne(p))
}\\

\inferrule{
    \text{add}(m;n;p) \quad \text{succ}(p;q)
}{
    \text{add}(\twiceOne(m);\twiceOne(n);\twice(q))
}
\end{mathpar}

\end{enumerate}

\end{document}
