\documentclass[12pt]{article}
\setlength{\oddsidemargin}{0in}
\setlength{\evensidemargin}{0in}
\setlength{\textwidth}{6.5in}
\setlength{\parindent}{0in}
\setlength{\parskip}{\baselineskip}

\usepackage{amsmath,amsthm,amsfonts,amssymb,fancyvrb}
\usepackage{mathpartir}

% Helpful macros
\newcommand{\mt}[1]{\ensuremath{\text{#1}}}
\newcommand{\isA}[2]{\ensuremath{#1 \; #2}}
\newcommand{\isANat}[1]{\isA{#1}{\mt{nat}}}
\newcommand{\zero}{\mt{zero}}
\newcommand{\mySucc}[1]{\mt{succ}(#1)}


%% Judgments


\begin{document}

PFPL \hfill Chapter 2 Exercise\\
Your Name

\hrulefill

\subsection*{Questions}

\begin{enumerate}
\item[2.1] Give an inductive definition of the judgment $\text{max}(m;n;p)$ where $\isANat{m}, \isANat{n}, \isANat{p}$, with the meaning that $p$ is the larger of $m$ and $n$. Prove that every $m$ and $n$ are related to a unique $p$ by this judgment.

TODO

\item[2.5] Give an inductive definition of the \textit{binary natural numbers}, which are either zero, twice a binary number, or one more than twice a binary number. The size of such a representation is logarithmic, rather than linear, in the natural number it represents.

TODO

\item[2.6] Give an inductive definition of addition of binary natural numbers as defined in Exercise \textbf{2.5}.\\ \textit{Hint:} Proceed by analyzing both arguments to the addition, and make use of an auxiliary function to compute the successor of a binary number. Hint: Alternatively, define both the sum and the sum-plus-one of two binary numbers mutually recursively.

TODO

\end{enumerate}

\subsection*{Appendix}
Here is an example latex typesetting of an inductive definition. You can copy/paste and modify this code. After finishing all the exercises, feel free to delete this section.

\begin{mathpar}
\inferrule{
}{
    \isANat{\zero}
}

\inferrule{
    \isANat{n}
}{
    \isANat{\mySucc{n}}
}
\end{mathpar}


\end{document}
