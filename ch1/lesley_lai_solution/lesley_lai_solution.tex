\documentclass[12pt]{article}
\setlength{\oddsidemargin}{0in}
\setlength{\evensidemargin}{0in}
\setlength{\textwidth}{6.5in}
\setlength{\parindent}{0in}
\setlength{\parskip}{\baselineskip}

\usepackage{amsmath,amsthm,amsfonts,amssymb,fancyvrb}
\usepackage{mathpartir}

\begin{document}

PFPL \hfill Chapter 1 Exercise\\
Lesley Lai

\hrulefill

\begin{enumerate}
\item[1.1] Prove by structural induction on abstract syntax trees that if $\mathcal{X} \subseteq \mathcal{Y}$, then $\mathcal{A}[\mathcal{X}] \subseteq \mathcal{A}[\mathcal{Y}]$\\


\begin{proof}

\textit{Any variables in $\mathcal{A}[\mathcal{X}]$ is also a valid variable in $\mathcal{A}[\mathcal{Y}]$ since $\mathcal{X} \subseteq \mathcal{Y}$. And for the inductive case, Let $o(a_1, \cdots, a_n)$ be a valid AST in $\mathcal{A}[\mathcal{X}]$. If $a_1 \in \mathcal{A}[\mathcal{X}], \cdots a_n \in \mathcal{A}[\mathcal{X}]$, by induction hypothesis $a_1 \in \mathcal{A}[\mathcal{Y}], \cdots a_n \in \mathcal{A}[\mathcal{Y}]$, which means that $o(a_1, \cdots, a_n)$ is also a valid AST in $\mathcal{A}[\mathcal{Y}]$. Thus, by structural induction, if $\mathcal{X} \subseteq \mathcal{Y}$, then $\mathcal{A}[\mathcal{X}] \subseteq \mathcal{A}[\mathcal{Y}]$.}

\end{proof}


\end{enumerate}

\end{document}
